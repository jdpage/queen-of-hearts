\documentclass[12pt,letter]{article}
\usepackage{amssymb, amsmath}
\usepackage[margin=1in]{geometry}

\title{The Queen of Hearts}
\author{Jonathan David Page}
\date{April 21, 2015}

\begin{document}
\maketitle

\section{Program Notes}

\emph{The Queen of Hearts} is a piece for solo voice. Based on \emph{Alice in Wonderland} by Lewis Carroll, a.k.a. mathematician Charles Dodgson, it re-imagines the dialogue of the Queen of Hearts. Taking the themes of the book to extremes, on the surface it makes only fleeting sense, while being built around a rigid mathematical structure based on Julia sets generalized to a finite ring of Gaussian integers.

Sections of the score may not be reused, and once performed once must be regenerated before another performance. A subset of the score to perform is to be selected by a simple audience participation activity involving playing cards. The performer is not to be shown the score in advance, and must rehearse using stale copies.

\section{Technical Notes}

Take the ring $\mathbb{Z}_n$, where the elements of the set are the non-negative integers less than $n$ and the operations $+$ and $\cdot$ are taken mod $n$. Let $\mathbb{Z}_n[i]$ be said ring adjoined with the element $i$ where $i^2 = -1$. Given an element $c \in \mathbb{Z}_n[i]$, define a class of sequences $(z_n)$ where $z_{n+1} = z_n^2 + c$. Due to the finite nature of the ring, such a sequence must have cycles.

The backbone of the piece was constructed by taking all such cycles in the ring $\mathbb{Z}_{19}[i]$ of length $\ge 30$ and taking the magnitude and angle of each element of the ring. The magnitude was mapped to sentence length, while the angle was mapped to the mood of the sentence. Each cycle produces one section of the score.

The sentences themselves were generated using a Markov process generated by doing a statistical analysis of the dialogue of the Queen of Hearts in \emph{Alice in Wonderland}.

Due to the cyclic nature of each section, the performer may start anywhere in the section, looping back around to the beginning, as long as they perform the entire section exactly once.

\section{About the Composer}

Jonathan David Page is a senior in mathematics at North Carolina State University. He was brought up on classical music, and played trumpet in high school. His favorite book is actually \emph{Through the Looking Glass}, not \emph{Alice in Wonderland}. 

\end{document}